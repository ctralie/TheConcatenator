% -----------------------------------------------
% Template for ISMIR Papers
% 2023 version, based on previous ISMIR templates

% Requirements :
% * 6+n page length maximum
% * 10MB maximum file size
% * Copyright note must appear in the bottom left corner of first page
% * Clearer statement about citing own work in anonymized submission
% (see conference website for additional details)
% -----------------------------------------------

\documentclass{article}
\usepackage[T1]{fontenc} % add special characters (e.g., umlaute)
\usepackage[utf8]{inputenc} % set utf-8 as default input encoding
\usepackage{ismir,amsmath,cite,url}
\usepackage{graphicx}
\usepackage{color}


\usepackage{lineno}
\linenumbers

\graphicspath{{../Figures/}}

% Title. Please use IEEE-compliant title case when specifying the title here,
% as it has implications for the copyright notice
% ------
\title{The Concatenator: A Bayesian Approach To Real Time Concatenative Musaicing}

% Note: Please do NOT use \thanks or a \footnote in any of the author markup

% Single address
% To use with only one author or several with the same address
% ---------------
%\oneauthor
% {Names should be omitted for double-blind reviewing}
% {Affiliations should be omitted for double-blind reviewing}

% Two addresses
% --------------
\twoauthors
  {First author} {School \\ Department}
  {Second author} {Company \\ Address}

% Three addresses
% --------------\input{ISMIR2021_paper.tex}

%\threeauthors
%  {First Author} {Affiliation1 \\ {\tt author1@ismir.edu}}
%  {Second Author} {\bf Retain these fake authors in\\\bf submission to preserve the formatting}
%  {Third Author} {Affiliation3 \\ {\tt author3@ismir.edu}}

% Four or more addresses
% OR alternative format for large number of co-authors
% ------------
%\multauthor
%{First author$^1$ \hspace{1cm} Second author$^1$ \hspace{1cm} Third author$^2$} { \bfseries{Fourth author$^3$ \hspace{1cm} Fifth author$^2$ \hspace{1cm} Sixth author$^1$}\\
%  $^1$ Department of Computer Science, University , Country\\
%$^2$ International Laboratories, City, Country\\
%$^3$  Company, Address\\
%{\tt\small CorrespondenceAuthor@ismir.edu, PossibleOtherAuthor@ismir.edu}
%}

% For the author list in the Creative Common license, please enter author names. 
% Please abbreviate the first names of authors and add 'and' between the second to last and last authors.
\def\authorname{F. Author, S. Author}

% Optional: To use hyperref, uncomment the following.
\usepackage[bookmarks=false,pdfauthor={\authorname},pdfsubject={\papersubject},hidelinks]{hyperref}
% Mind the bookmarks=false option; bookmarks are incompatible with ismir.sty.

\sloppy % please retain sloppy command for improved formatting

\newcommand{\ChrisEdit}[1]{\textcolor{red}{(#1)}}
\newcommand{\BenEdit}[1]{\textcolor{blue}{(#1)}}

\begin{document}

%
\maketitle
%

\begin{abstract}

\end{abstract}

\section{Introduction}

Speed/accessibility will impact speed of research and innovation

\section{Related Work}
This work builds on a long history of work in Bayesian inference and particle filters, concatenative synthesis, and applied nonnegative matrix factorization (NMF).

From an artistic point of view, the most similar technique to ours is Driedger et al.'s 2015 ``Let It Bee'' concatenative musaicing technique \cite{driedger2015let}, which uses NMF to learn activations of spectral templates so that their combination will match a target spectrogram.
\BenEdit{TODO: Talk about Zero Point and Edenic}


We now detail some of the mathematics of Driedger et al.'s technique, as we borrow a few ideas in our work.  Driedger et al. learn $H$ in the equation $V \approx WH$, where $V$ is an $M \times T$ target spectrogram with $M$ frequency bins and $T$ times, $W$ is an $M \times N$ set of $N$ spectral templates that are treated as fixed, and $H$ is a matrix of $N \times T$ learned activations.  For instance, $W$ could be the spectral windows of a collection of buzzing bees and $V$ could be an excerpt from The Beatles' ``Let It Be'' (hence the title).  Driedger et al. use the Kullback-Liebler (KL) divergence loss, an instance of the more general $\beta$-divergence \cite{buch2017nichtnegativematrixfaktorisierungnutzendesklangsynthesensystem}, to measure the goodness of fit of $WH$ to $V$.  This loss function is 

\begin{equation}
\label{eq:klloss}
D(V || WH) = V \odot \log \left( \frac{V}{WH} \right) - V + WH
\end{equation}

where $\odot$, $/$, $+$, and $-$ are all applied element-wise.  As Lee/Seung show, choosing the right step size turns gradient descent of Equation~\ref{eq:klloss}, with respect to $W$ and $H$, into {\em multiplicative update rules} that guarantee monotonic convergence.  Driedger et al. keep $W$ fixed to force the final audio to use exact copies of the templates, so only the update rule for $H$ is relevant.  At iteration $\ell$, this is:

\begin{equation}
\label{eq:klhgrad}
H_{kt}^{\ell} \gets H_{kt}^{\ell-1} \left( \frac{ \sum_{n} W_{nk} V_{nt} / (WH^{\ell})_{nt} }{ \sum_{n} W_{nk} } \right)
\end{equation}

Crucially, though, Driedger et al. note that the update rules in Equation~\ref{eq:klhgrad} alone will lose the timbral character of the templates in $W$.  They hence disrupt ordinary KL gradient descent by performing several increasingly impactful modifications to $H$ before Equation~\ref{eq:klhgrad} in each step, which are eventually set in stone after $L$ total iterations.  First, they avoid repeated activations to avoid a ``jittering'' effect, allowing a particular activation $k$ to only exist once in some iterval based on where it's the strongest:
\begin{equation}
    \label{eq:driedgerrepeated}
    (H_r)_{kt}^{\ell} \gets \left\{ \begin{array}{cc} H^{\ell-1}_{kt} & H^{\ell-1}_{kt} > H^{\ell-1}_{ks}, |t - s| \leq r \\ H^{\ell-1}_{kt} (1 - \frac{\ell+1}{L}) & \text{otherwise}  \end{array} \right\}
\end{equation}
for some chosen $r$.  They also promote sparsity in a similar way by shrinking all but the top $p$ activations in each column of $H_r$ to create $H_p^{\ell}$.  Finally, they add a step to encourage {\em time continuous activations} by doing ``diagonal enhancement,'' or by doing a windowed sum down each diagonal of $H_p$, assuming the columns of $W$ are also in a time order\footnote{A multiscale music structure analysis work uses a similar idea \cite{mcfee2014analyzing}}:

\begin{equation}
    \label{eq:driedgertimecontinuous}
    (H_c)_{kt}^{\ell} = \sum_{i=-c}^c (H_p)^{\ell}_{k+i, t+i}
\end{equation}

Since this encourages the algorithm to mash up chunks of $W$ in a time order, it effectively encourages sound grains from the templates than the length of a single window that ordinary NMF would take.  Finally, Equation~\ref{eq:klhgrad} is applied to $H_c^{\ell}$ to obtain $H^{\ell}$.

Overall, these disruptions remove the guarantee that Equation~\ref{eq:klloss} will be minimized, or that it will even monotonically decrease, but Driedger et al.'s brilliant insight is that the loss function is merely a guide to choose reasonable activations; a sub-optimal fit leaves room to better preserve timbral characteristics.  We take a similar perspective.



\subsubsection{Beyond Driedger}
Interestingly, the idea of spectrogram decomposition used for concatenative musaicing goes back to the work of \cite{burred2013cross}.  Beyond that, the authors of \cite{buch2017nichtnegativematrixfaktorisierungnutzendesklangsynthesensystem} provide some improvements to Driedger et al.'s technique, including sidestepping phase retrieval, which we exploit in our work.

One issue with Driedger et al.'s technique is the sources have to be augmented with pitch shifts to span additional pitches in the target, increasing memory consumption and runtime.  The authors of \cite{foroughmand2017multi, aarabi2018music} sidestep this by using 2D deconvolutional NMF \cite{schmidt2006nonnegative}.  Specifically, their technique uses the Constant-Q transform, where pitch shifts are modeled as constant shifts of the activations instead of the templates, saving memory.  The other added activation dimension models time history and time shifts to avoid the need for the diagonal enhancement of Equation~\ref{eq:driedgertimecontinuous}.  Interestingly, the authors apply 2D NMF to both the source material and the target, so they do not preserve the original sound grains.  However, for our preferred style, we want to take the source grains exactly as they are.



Schwarz created an offline concatenative synthesis system dubbed ``Caterpillar'' that uses the Viterbi algorithm \cite{schwarz2000system}, which was later approximated with a real time system, "CataRT" that uses a greedy approach instead of the Viterbi algorithm \cite{schwarz2006real, schwarz2008principles}.  Simon's ``audio analogies'' is quite similar \cite{simon2005audio}, but instead of a user controlled traversal through timbral space, they use features from some source (e.g. midi audio) to guide synthesis to a target with a different timbre (e.g. real audio of someone playing a trumpet).  \BenEdit{Need some additional references on granular synthesis}.

Caterpillar and audio analogies are both {\em Bayesian in nature}, where the {\em hidden state} is the template to concatenate, and the ``observation'' is a user-controlled trajetory or features from a source timbre, respectively.  The prior transition probabilities are based on temporal continuity.  However, they use the Viterbi algorithm, which is computationally intensive and which needs all time history, so it cannot be applied in real time.  A particle filter is an alternative for real time Bayesian inference \ChrisEdit{find particle filter citations}.  Particle filters have found use in other real time MIR applications such as multi-pitch tracking \cite{duan2011state} and beat tracking \cite{heydari2021don}.  

\section{The Concatenator}

\subsection{Bayesian Formulation}
\label{sec:bayesian}

    The NMF technique of Driedger is not suitable for realtime applications, both because the gradient update rules of Equation~\ref{eq:klhgrad} are too slow, and because the equations to promote repeated activations and time-continuous at each entry of $H$ require knowledge of all activations in $H$, including future activations.  We instead turn to a Bayesian formulation, where we view the target spectrogram $V$ in Equation~\ref{eq:klloss} as an observation, and the hidden states are the activations of the $N$ templates in the corpus spectrogram $W$ at each time.  We then use a particle filter to efficiently infer the hidden activations (Section~\ref{sec:sampling}).
    
    To control polyphony, we model each activation $\vec{h_t}$ as a $p$-sparse vector; that is:

    \begin{equation} 
        \vec{h_t}[k] \in \left\{0, 1, ..., N-1\right\}, k = 0, 1, ..., p-1 
    \end{equation}

    Given associated nonnegative weights $\vec{\alpha_t}[k]$ for each activation, the approximation at time $t$ is 
    
    \begin{equation}
        WH_{nt} = \sum_{k} \vec{\alpha_t}[k]  W_{n, \vec{h_t}[k]}
    \end{equation}

    To infer $\vec{\alpha_t}$, we apply the update rules of Equation~\ref{eq:klhgrad} for a pre-specified number $L$ of iterations, using the corresponding columns of $W$ prescribed by $\vec{h_t}$:
    
    \ChrisEdit{TODO: Add equation}
    
    We use Equation~\ref{eq:klloss} to measure the spectral fit to the target, though as in Driedger et al. \cite{driedger2015let} (Equation~\ref{eq:driedgertimecontinuous}), we are willing to sacrifice fit to take longer grains from the corpus $W$.  To that end, we define the prior {\em state transition probability} in the as a Factorial Hidden Markov Model (FHMM) \cite{ghahramani1995factorial}, where components of each activation transition independently from each other:

		\begin{equation}
        	p(\vec{h_t} = \vec{b} | \vec{h_{t-1}} = \vec{a}) = \prod_{k=0}^{p-1} \left\{  \begin{array}{cc}  p_d & b[k] = a[k]+1  \\ \frac{1-p_d}{N-1} & \text{otherwise} \end{array} \right\}
		\end{equation}
        FHMMs have been used in similar ways in the context of multi-pitch tracking \cite{wohlmayr2010probabilistic}. 
        
        To track the target in a Bayesian framework, we devise an {\em observation probability} based on $\vec{\alpha_t}$


\subsection{Sampling Algorithm}
\label{sec:sampling}

Discuss particle filter and synthesis details


\section{Mathematical Analysis}

\subsection{Picking Good Activations}

In practice, a limited number of particles are surprisingly effective at picking up on activations that fit the target well, which we explain with a simple probabilistic argument.  Given a corpus with $N$ sound grains (including pitch shifts) and $P$ particles that each capture $p$ activations, suppose that we have a hypothetical ``ideal particle'' $\hat{h}$ with the $p$ best activations that work together to match a target.  Since activations adapt to the target when they are randomly resampled with probability $p_d$, the probability of jumping to at least one of the top $k$ activations of $\hat{h}$ at a particular timestep is 

\begin{equation}
    \label{eq:timeadjacentprob}
    1 - \left( p_d + (1-p_d) \frac{(N-2-k)}{N} \right)^{pP}
\end{equation}

This probability is small (about 0.058, 0.077, 0.095 for $k = 1, 2, 3$ using $N=10000$ and $P=2000$).  However, the hop length is quite small (1024/44100 ~= 23 milliseconds), so if we allow the right activation to be chosen slightly before or slightly after the exact time it should be chosen, we wouldn't notice it and the result would still sound good.  Furthermore, for each activation in $\hat{h}$, there are usually several activations in the corpus that sound acceptably similar.  Let $\delta$ be the amount of wiggle room before or after in time for choosing the best activations, and let $w$ be a factor of acceptible activations nearby (e.g. $w=3$ would allow the us to consider each activation in $\hat{x}$ and its closest two in the corpus).  Equation~\ref{eq:timeadjacentprob} is then updated as:

\begin{equation}
    \label{eq:timeadjacentprobmodified}
    1 - \left( p_d + (1-p_d) \frac{(N-2-wk)}{N} \right)^{(2 \delta +1)pP}
\end{equation}

For example, for $\delta=5$ (about 116 milliseconds of deviation) and $w = 5$, the probabilities are 0.786, 0.929, 0.976 for $k=1, 2, 3$, respectively.  These probabilities all degrade when $N$ gets larger for a larger corpus, but in that case, it is likely that the acceptable $w$ is also larger.

We also note that once one of the particles catches on to a good activation, it is promoted with a high weight and gets carried on with {\em time-continuous activations}.  This is similar to how the ``patch match'' technique in computer graphics \cite{Barnes:2009:PAR, Barnes:2010:TGP} computes nearest neighbors of many nearby patches by starting with a random initializations of nearest neighbors.  Similarly, in that case, chances are that at least one of the patches will be matched to a correct neighbor, and {\em spatially adjacent} patches correct their nearest neighbors to to neighbors at the corresponding offsets \cite{Barnes:2009:PAR}.

\subsection{Diagonal Enhancement}

Talk about negative binomial, show example plot of diagonal distribution for different $p_d$


\section{Evaluation}

\subsection{Quantitative Evaluation}
Show fit, distribution of diagonal lengths, distribution of activation changes, and repeated activation lengths for this algorithm compared to Driedger algorithm over the Tzanetakis dataset


\subsection{Qualitative Evaluation}

Talk about Ben's ``obstacle course''

% For bibtex users:
\bibliography{ISMIRtemplate}

% For non bibtex users:
%\begin{thebibliography}{citations}
% \bibitem{Author:17}
% E.~Author and B.~Authour, ``The title of the conference paper,'' in {\em Proc.
% of the Int. Society for Music Information Retrieval Conf.}, (Suzhou, China),
% pp.~111--117, 2017.
%
% \bibitem{Someone:10}
% A.~Someone, B.~Someone, and C.~Someone, ``The title of the journal paper,''
%  {\em Journal of New Music Research}, vol.~A, pp.~111--222, September 2010.
%
% \bibitem{Person:20}
% O.~Person, {\em Title of the Book}.
% \newblock Montr\'{e}al, Canada: McGill-Queen's University Press, 2021.
%
% \bibitem{Person:09}
% F.~Person and S.~Person, ``Title of a chapter this book,'' in {\em A Book
% Containing Delightful Chapters} (A.~G. Editor, ed.), pp.~58--102, Tokyo,
% Japan: The Publisher, 2009.
%
%
%\end{thebibliography}

\end{document}

